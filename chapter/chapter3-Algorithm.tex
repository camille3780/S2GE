\begin{ZhChapter}

\chapter{Proposed Algorithm}
In this chapter, we will introduce the architecture and flowchart of our algorithm. Next, we will provide an in-depth explanation of our proposed algorithm, including the underlying concepts and the rationale behind our approach. Subsequently, we will present a detailed description of our software and hardware specifications, as well as the setup procedures. Finally, we will execute the experiment and describe how we implemented our concept and successfully completed the entire experiment.
\section{The Architecture of Algorithm} %Chapter 3.1
\begin{figure*}[htbp]
    \centering
    \includegraphics[width = 1\textwidth]{image/FlowChart.png}
    \caption{FlowChart for Efficiency-based GP}
    \label{fig: FlowChart}
\end{figure*}
% \section{Experimental Environment Setup}

% \subsection{Software Setup}
% In our software setup, we run our code in Ubuntu 24.04. The Python version we employ is 3.8, and the Pytorch version is X. Within Python and Pytorch, we utilize the following packages: Some package.

% \subsection{Hardware Setup}
% In our hardware setup, we run our code on an AMD Ryzen 5 5600X 6-Core Processor. The system is equipped with 32GB of memory, operating at a frequency of DDR4-3600MHz. Additionally, we utilize an NVIDIA GeForce RTX 3060 Ti GPU for our computational needs.

\section{Main Method}
% \documentclass{article}
Based on the architecture we introduced on section 3.1, the pseudo code of this algorithm is presented as below.
\begin{algorithm}
    \caption{Efficiency-based GP to generate loss function}\label{alg:cap}
    \begin{algorithmic}
        \State Initialize: \space $P = \space 8$, $T =\space 12$, $G =\space P*3/4$, $N = \space G/2$, $C_r = \space0.5$, $M_{ST} = \space0.3$, $M_N = \space0.1$
        \State Randomly initialize population which include P trees
        \State Evaluate GP fitness function $F$ for each individual in the population
        \State Determine the best individual
        \While{generation number < $T$}
        \State $S$ = Sample tournament G $\sim$ Uniform (P)
        \State Select top $N$ trees from $G$ to form a set $S$
        \For{Individual in $S$}
        \If {$rand_1 < 0.5$}
        \State random\_individual = Randomly select a individual in $S$ except Individual
        \Else
        \State random\_individual = Randomly generated tree
        \EndIf
        \State generated\_child = Crossover(Individual, random\_individual)
        \If {$rand_2 < M_{ST}$}
        \State Apply subtree mutation to the generated\_child
        \EndIf
        \If {$rand_3 < M_N$}
        \State Apply one-point mutation to the generated\_child
        \EndIf
        \EndFor
        \State Evaluated GP fitness function $F$ for each generated child individual
        \State Update the best individual
        \State Select $P$ best trees from population to form a new population containing $P$ trees
        \EndWhile
    \end{algorithmic}
\end{algorithm}

\section{Implementation Details}
To implement our experiment, the hardware and software requirements are first defined in Section 3.3.1. Next, the environment setup steps are described in Section 3.3.2. Finally, the implementation steps are explained in Section 3.3.3.
\subsection{Requirements}

Table \ref{tab: softwareSpec} outlines our software requirements. Our code execution environment is Windows 11. We utilize Python version 3.9.18 and TensorFlow version 2.6.0. The specific packages used within Python and TensorFlow are detailed in Table \ref{tab: packageSpec}.

\begin{table*}[htbp]
\centering
\caption{Software requirements} \label{tab: softwareSpec}
\makebox[\linewidth][c]{
    \renewcommand\arraystretch{1.2}{
        \begin{tabular}{| l | c  c |}
        \hline
        Software Type & Software Name & Version \\
        \hline
        Operating System & Windows &  11 \\
        Programming Language & Python & 3.9.18\\
        \hline
        \end {tabular}
    }}
\end {table*}

\begin{table*}[htbp]
\centering
\caption{Python package requirements} \label{tab: packageSpec}
\makebox[\linewidth][c]{
    \renewcommand\arraystretch{1.2}{
        \begin{tabular}{| l | c  c |}
        \hline
        Package Name/Version & Imported from & License \\
        \hline
        tensorflow/2.6.0 & N/A & Apache License 2.0\\
        numpy/1.20.3 & N/A & modified BSD license\\
        cudnn/8.2.1 & N/A & N/A\\
        os & N/A & N/A\\
        copy  & N/A & N/A\\
        time  & N/A & N/A\\
        math  & N/A & N/A\\
        random & N/A & N/A\\
        datetime & datetime & N/A\\
        cmp\_to\_key & functools & N/A\\
        Sequential & tensorflow.keras.models & N/A\\
        Dense & tensorflow.keras.layers & N/A\\
        Flatten & tensorflow.keras.layers & N/A\\
        Dropout & tensorflow.keras.layers & N/A\\
        UpSampling2D & tensorflow.keras.layers & N/A\\
        BatchNormalization & tensorflow.keras.layers & N/A\\
        InceptionV3 & tensorflow.keras.applications.inception\_v3 & N/A\\
        preprocess\_input & tensorflow.keras.applications.inception\_v3 & N/A\\
        ReduceLROnPlateau & tensorflow.keras.callbacks & N/A\\
        to\_categorical & tensorflow.keras.utils & N/A\\
        train\_test\_split/1.5.2 & sklearn.model\_selection & N/A\\
        ImageDataGenerator & tensorflow.keras.preprocessing.image & N/A\\
        np\_config & tensorflow.python.ops.numpy\_ops & N/A\\
        \hline
        \end {tabular}
    }}
\end {table*}

Table \ref{tab: hardwareSpec} presents our hardware requirements. The computer we used to execute our code is equipped with a 13th Gen Intel(R) Core(TM) i7-13700 CPU, 32GB of memory operating at a frequency of DDR5-5600MHz, and an NVIDIA GeForce RTX 4060 GPU to meet our computational needs.

\begin{table*}[htbp]
\centering
\caption{Hardware requirements} \label{tab: hardwareSpec}
\makebox[\linewidth][c]{
    \renewcommand\arraystretch{1.2}{
        \begin{tabular}{| l | c |}
        \hline
        Hardware Type & Name \\
        \hline
        CPU & 13th Gen Intel(R) Core(TM) i7-13700 \\
        Memory & DDR5-5600MHz 32 GB \\
        Graphic Card & NVIDIA GeForce RTX 4060 \\
        \hline
        \end {tabular}
    }}
\end {table*}

\subsection{Environment Setup}

The following section will describe how to properly set up the environment we used to conduct our experiment. First, we used Anaconda as our package management tool for the implementation. Therefore, you must first visit the Anaconda official website (\url{https://www.anaconda.com/download}) to download the software. As shown in figure \ref{fig: conda}, click the Download button to complete the download.

\begin{figure*}[htbp]
    \centering
    \includegraphics[width = 0.75\textwidth]{image/conda.jpg}
    \caption{Anaconda official website}
    \label{fig: conda}
\end{figure*}

After installation, we need to create a new environment to ensure the independence of all the packages. In Anaconda prompt, enter \verb|conda create --name environment-name| to create a new environment, as shown in figure \ref{fig: createEnv}. Next, enter \verb|conda activate environment-name| to activate the environment just like the first prompt entered in figure \ref{fig: tensorflow}.

\begin{figure*}[htbp]
    \centering
    \includegraphics[width = 0.75\textwidth]{image/createEnv.png}
    \caption{Conda prompt for creating new environment}
    \label{fig: createEnv}
\end{figure*}

Since directly installing TensorFlow may result in the installation of incompatible versions of NumPy, we first use the command shown in figure \ref{fig: numpy} to download version 1.20.3 of  NumPy: \verb|conda install numpy=1.20|. Afterward, we can install the GPU version of TensorFlow by entering the following command in the Anaconda prompt: \verb|conda install tensorflow-gpu|, as shown in figure \ref{fig: tensorflow}. Finally, we need to download Scikit-learn to split the dataset during the model training process. As shown in figure \ref{fig: sklearn}, enter the following command to complete the final installation step: \verb|conda install scikit-learn|.

\begin{figure*}[htbp]
    \centering
    \includegraphics[width = 0.75\textwidth]{image/numpy.png}
    \caption{Install numpy in conda environment}
    \label{fig: numpy}
\end{figure*}

\begin{figure*}[htbp]
    \centering
    \includegraphics[width = 0.75\textwidth]{image/tensorflow.png}
    \caption{Install tensorflow in conda environment}
    \label{fig: tensorflow}
\end{figure*}

\begin{figure*}[htbp]
    \centering
    \includegraphics[width = 0.75\textwidth]{image/sklearn.png}
    \caption{Install scikit-learn in conda environment}
    \label{fig: sklearn}
\end{figure*}

\subsection{Implementation Steps}
% \subsection{}
% 定義定義定義定義定義定義\cite{latex2e},定義定義定義定義,定義定義定義定義定義定義定義定義定義定義,定義定義。

% \begin{table*}[htbp]
%     \centering
%     \caption{表格範例標題} \label{tab: complexity}
%     \makebox[\linewidth][c]{
%     \renewcommand\arraystretch{1.2}{
%         \begin{tabular}{| l | c  c  c  c |}
%         \hline
%         Protocol & $P$ & $CS_1$ & $CS_2$ & $RG$ \\
%         \hline
%         MSSMul & $O(1)$, $O(1)$, N/A & $O(n-t)$, $O(n)$, $O(1)$ & $O(n-t)$, $O(n)$, N/A & $O(1)$, $O(n)$, $O(n)$ \\
%         SC & $O(1)$, $O(1)$, N/A & $O(n-t)$, $O(n)$, $O(1)$ & $O(n-t)$, $O(n)$, N/A & $O(1)$, $O(n)$, $O(n)$ \\
%         \hline 
%         \end {tabular}
%     }}
% \end {table*}

% \section{模型說明(小標)}

% 說明說明說明說明,說明說明說明說明說明說明說明說明說明說明說明說明,說明說明說明說明說明說明說明說明。

% \begin{figure*}[htbp]
%     \centering
%     \includegraphics[width = 0.5\textwidth]{image.jpeg}
%     \caption{Cool train station}
%     \label{fig: image}
% \end{figure*}

\end{ZhChapter}