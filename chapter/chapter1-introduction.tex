\begin{ZhChapter}
    \chapter{Introduction}
    Driven by the rapid advancement of digital transformation and smart infrastructure, the \textbf{Internet of Things (IoT)} has emerged as a cornerstone of next-generation information technology. Through the integration of sensors, embedded devices, communication modules, and platform software, IoT enables physical objects to communicate in real time and generate massive volumes of data. These data streams support a broad range of applications—such as smart manufacturing, intelligent transportation, remote healthcare, and smart homes—yielding substantial economic and societal value~\cite{atzori2010internet}.

    However, as the number of connected devices increases and deployment scenarios become more complex, IoT systems face unprecedented cybersecurity challenges. Many IoT devices are resource-constrained, infrequently updated, and difficult for users to manage. With limited encryption and a lack of monitoring mechanisms, these devices become prime targets for cyber intrusions and attacks. Effectively identifying abnormal behaviors and hidden threats in IoT network traffic has therefore become a pressing research priority.

    Furthermore, existing intrusion detection technologies often struggle to adapt to evolving threats. While deep learning approaches such as Word2Vec and Transformer-based models~\cite{devlin2018bert- vaswani2017attention} have demonstrated semantic learning capabilities, they also introduce critical drawbacks: large vocabulary requirements, high computational complexity, and limited flexibility in dynamic or resource-constrained environments. 

    To address these limitations, we propose \textbf{S2GE-NIDS} (Structured Semantics and Generation Embedded Network Intrusion Detection System)—a lightweight, interpretable anomaly detection framework designed for IoT environments. S2GE-NIDS combines hash-based token embedding with a multi-layer perceptron (MLP) model and introduces a linked-list mechanism to mitigate hash collisions inherent to non-cryptographic hash functions such as MurmurHash3~\cite{appleby2011murmurhash}. This design enables efficient feature encoding while avoiding the need to maintain a large vocabulary.

    In our approach, network packets are first transformed into semantic tokens and encoded using hash-based indexing. The resulting embedding vectors are concatenated into a single, fixed-length semantic vector, which is processed by an MLP and projected near a learned semantic center. Any significant deviation from this center—measured by Mahalanobis distance—is classified as a potential anomaly~\cite{liu2020anomaly}. 

    The proposed S2GE-NIDS framework offers several key advantages over conventional intrusion detection systems. First, it eliminates the need for manual feature engineering and vocabulary maintenance by using a hash-based embedding approach, where field-value pairs are directly encoded into semantic vectors without relying on predefined lookup tables. This design greatly simplifies the preprocessing pipeline and enhances scalability. Second, the model provides a mathematically interpretable anomaly scoring mechanism by integrating Mahalanobis distance, which quantifies how far a sample deviates from the learned distribution of normal behavior. This not only improves detection accuracy but also enables explainable results. Third, the system is lightweight and highly efficient, relying on simple MLP-based encoding instead of complex deep architectures, making it well-suited for deployment in real-time or resource-constrained environments such as edge devices in IoT networks. Lastly, its generalized tokenization strategy allows for wide applicability across diverse packet structures, further improving its adaptability and robustness in various network scenarios.

    The structure of this paper is as follows: Chapter 2 is the relevant background knowledge about S2GE-NIDS (Structured Semantics and Generation Embedded Network Intrusion Detection System). Chapter 3 introduces the architecture and methodology of the proposed S2GE-NIDS framework, presenting each module and its rationale in detail. Chapter 4 presents the experimental setup, evaluation metrics, and results on two benchmark datasets, as well as interpretability demonstrations. Chapter 5 summarizes the main findings, limitations, and directions for future research.
      
    

\end{ZhChapter}