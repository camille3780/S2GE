\begin{ZhChapter}
    \chapter{Introduction}
    Driven by the rapid advancement of digital transformation and smart infrastructure, the Internet of Things (IoT) has emerged as a cornerstone of next-generation information technology. Through the integration of sensors, embedded devices, communication modules, and platform software, IoT enables physical objects to communicate in real time and generate massive volumes of data. These data streams support a broad range of applications, such as smart manufacturing, intelligent transportation, remote healthcare, and smart homes, yielding substantial economic and societal value~\cite{atzori2010internet}.

    However, as the number of connected devices increases and deployment scenarios become more complex, IoT systems face unprecedented cybersecurity challenges. Many IoT devices are resource-constrained, infrequently updated, and difficult for users to manage. With limited encryption and a lack of monitoring mechanisms, these devices become prime targets for cyber intrusions and attacks. Effectively identifying abnormal behaviors and hidden threats in IoT network traffic has therefore become a pressing research priority.

    Furthermore, existing intrusion detection technologies often struggle to adapt to evolving threats. While deep learning approaches such as Word2Vec~\cite{sugathadasa2017synergistic} and Transformer-based models~\cite{han2022survey} have demonstrated semantic learning capabilities, they also introduce critical drawbacks: large vocabulary requirements, high computational complexity, and limited flexibility in dynamic or resource-constrained environments.

    To address these limitations, we propose Structured Semantics and Generation Embedded Network Intrusion Detection System (S2GE-NIDS) a lightweight, interpretable anomaly detection framework designed for IoT environments. S2GE-NIDS combines hash-based token embedding with a multi-layer perceptron (MLP) model and introduces a linked-list mechanism to mitigate hash collisions inherent to non-cryptographic hash functions such as MurmurHash3~\cite{yamaguchi2013hardware}. This design enables efficient feature encoding while avoiding the need to maintain a large vocabulary.

    In our approach, network packets are first transformed into semantic tokens and encoded using hash-based indexing. The resulting embedding vectors are concatenated into a single, fixed-length semantic vector, which is processed by an MLP and projected near a learned semantic center. Any significant deviation from this center measured by Mahalanobis distance is classified as a potential anomaly~\cite{liu2020anomaly}.

    The proposed S2GE-NIDS framework offers several advantages over conventional intrusion detection systems. First, it uses a hash-based embedding approach; The propose of the hash table in our method is efficiently searching data and storage and the combination of double hashing~\cite{guibas1976analysis} and linked list~\cite{dietz1982maintaining} serial nodes can compress and pre-allocate to save index space and reduce dynamic memory configuration costs. In addition, a fixed-size index after modulo operation is used to ensure that the size of the embedded table is controllable, taking into account both storage space and hashing uniformity. This strategy has better space complexity than directly hashing the entire key-value pair.

    Second, the model provides a mathematically interpretable anomaly scoring mechanism by integrating Mahalanobis distance, which quantifies how far a sample deviates from the learned distribution of normal behavior. This not only improves detection accuracy but also enables explainable results.

    Third, the system is lightweight and highly efficient, relying on simple MLP-based encoding instead of complex deep architectures \cite{naveed2023comprehensive}, making it well-suited for deployment in real-time or resource-constrained environments such as edge devices in IoT networks. Therefore, this double hashing architecture not only effectively optimizes the hash function combination, makes the hash more uniform, and reduces the chain length. It also improves the stability and efficiency of hash embedding, laying a good foundation for the subsequent anomaly detection model based on semantic vectors.


    The structure of this paper is organized as follows. Chapter 2 provides the background knowledge related to S2GE-NIDS. Chapter 3 presents the architecture and methodology of the proposed framework, detailing the design and each module. Chapter 4 describes the implementation setup and steps, Chapter 5 provides the experimental results. Finally, Chapter 6 concludes and outlines for future research.



\end{ZhChapter}