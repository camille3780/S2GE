\begin{ZhChapter}
    \chapter{Related Work}
    This section will introduce the relevant basic knowledge, including existing IoT network intrusion detection methods, Tokenization, Hash Embedding and language tags.

    \section{Network Intrusion Detection System in IoT}
    In recent years, the proliferation of Internet of Things (IoT) devices has led to an increased focus on developing effective network intrusion detection systems (NIDS) tailored to the specific characteristics of IoT environments. Various approaches have been proposed to address the challenges associated with high-volume, heterogeneous network traffic, constrained device capabilities, and evolving attack patterns. Kharoubi et al.~\cite{kharoubi2025nidscnn} proposed NIDS-DL-CNN, a convolutional neural network (CNN)-based detection system designed for IoT security. By applying CNN layers to extract spatial features from traffic data, the model achieved high classification performance on datasets such as CICIoT2023 and CICIoMT2024. The authors demonstrated that their method achieved excellent precision and recall in both binary and multi-class scenarios. However, a notable limitation of the CNN-based approach lies in its inability to fully capture temporal dependencies across packet sequences, and its reliance on supervised learning requires extensive labeled datasets. Ashraf et al.~\cite{ashraf2025inids} introduced a real-time intrusion detection system (INIDS) based on traditional machine learning classifiers applied to the BoT-IoT dataset. The study compared seven algorithms, including Random Forest, Artificial Neural Networks (ANN), and Support Vector Machines. Their results showed that Random Forest and ANN achieved the highest accuracy and robustness among all tested classifiers. Despite its efficiency, the INIDS system was highly dependent on manual feature engineering and lacked adaptability to novel threats, which are critical in fast-evolving IoT environments. Elrawy et al.~\cite{elrawy2018survey} conducted a comprehensive survey of intrusion detection methodologies in IoT-based smart environments, categorizing techniques according to architectural design (centralized vs. distributed), detection strategy (signature-based, anomaly-based, or hybrid), and system layer (perception, network, application). While the survey provided valuable insights and synthesized a broad range of IDS approaches, it lacked implementation evidence and empirical comparisons, limiting its utility for practical system design. Collectively, these studies highlight the trade-offs between detection performance, computational cost, and deployment feasibility. Deep learning models offer strong accuracy but demand computational resources, while traditional classifiers provide efficiency but often lack flexibility. In contrast, our proposed S2GE-NIDS framework leverages hash-based semantic embeddings and a lightweight MLP, offering a balanced approach that combines scalability, interpretability, and effectiveness in detecting network anomalies in resource-constrained IoT environments.

    \begin{table*}[htbp]
        \centering
        \caption{Common Anomalous Features in IoT Network Traffic and Their Descriptions}
        \label{tab:iot_abnormal_features}
        \makebox[\linewidth][c]{
            \renewcommand\arraystretch{1.3}{
                \begin{tabular}{| l | p{12cm} |}
                \hline
                \textbf{Feature (with Reference)} & \textbf{Description} \\
                \hline
                Destination Port~\cite{tang2019deep} & Specific port targets (e.g., 22, 23, 80, 443) are often associated with attacks. Abnormal access to these ports may suggest behaviors such as scanning, DDoS, or brute-force intrusion. \\
                \hline
                Flow Duration~\cite{tharewal2022intrusion} & Extremely short or long connection durations within brief timeframes may signal scanning activity or data exfiltration. \\
                \hline
                Total Forward Packets~\cite{sharafaldin2018cicflowmeter} & Unusually high or low packet counts in one direction may indicate abnormal sessions or flooding behavior. \\
                \hline
                Packet Length~\cite{tang2016deep} & Anomalies in packet size—whether fixed, too long, or too short—often reflect malicious traffic like botnet propagation or worms. \\
                \hline
                Protocol Type~\cite{tharewal2022intrusion} & Sudden increases in uncommon protocols (e.g., ICMP, UDP) may reveal attempts to exploit protocol vulnerabilities or bypass filters. \\
                \hline
                Source IP / Destination IP~\cite{tavallaee2009kdd} & Repeated access from abnormal IP addresses, or sudden surges in novel IP sources, are indicative of scanning, spoofing, or DDoS activity. \\
                \hline
                Flow Bytes per Second~\cite{tang2016deep} & Sharp fluctuations—surges or drops—in flow byte rate may suggest DoS attacks or unauthorized data transfer. \\
                \hline
                TCP Flags~\cite{sharafaldin2018cicflowmeter} & Unusual combinations (e.g., SYN, FIN, RST) can indicate stealth scans or TCP-based flooding. \\
                \hline
                Number of Connections~\cite{tang2016deep} & A large number of new connections established by a single IP in a short time often reflects worm propagation or botnet coordination. \\
                \hline
                \end{tabular}
            }}
        \end{table*}
        
        

        

    \section{Tokenization}
    Following feature extraction, the next critical step is the tokenization process, which prepares network traffic data for semantic embedding. Each data record typically contains multiple fields—such as \texttt{Port}, \texttt{Protocol}, and \texttt{SrcIP}—representing structural and behavioral attributes of a network flow. To ensure consistency and distinguishability among features during embedding, we adopt a composite tokenization strategy that combines each field name with its corresponding value to form a unique token string. For instance, a sample token may take the form \texttt{Protocol:TCP} or \texttt{DstPort:443}.
    
    This strategy preserves the semantic association between field-value pairs without relying on predefined vocabularies, making it particularly suitable for dynamic and heterogeneous IoT environments. Each composite token is subsequently encoded using hash-based mapping techniques \ref{sec:hash_embedding}, thereby eliminating the need for extensive memory allocation or manually constructed token dictionaries. By treating each token as a self-contained semantic unit, this method also enhances the model’s ability to generalize to previously unseen feature combinations, ultimately improving both the adaptability and scalability of the proposed system~\cite{weinberger2009feature}.


    \begin{table*}[htbp]
        \centering
        \caption{Examples of Field-Value Tokenization in IoT Network Traffic}
        \label{tab:tokenization_examples}
        \makebox[\linewidth][c]{
            \renewcommand\arraystretch{1.2}{
                \begin{tabular}{| l | l |}
                \hline
                \textbf{Feature Field} & \textbf{Tokenized Representation} \\
                \hline
                Protocol = TCP & Protocol:TCP \\
                \hline
                Destination Port = 443 & DstPort:443 \\
                \hline
                Source Port = 80 & SrcPort:80 \\
                \hline
                Source IP = 192.168.0.1 & SrcIP:192.168.0.1 \\
                \hline
                Flow Duration = 120000 & FlowDuration:120000 \\
                \hline
                Payload Bytes = 56 & PayloadBytes:56 \\
                \hline
                Packet Count = 10 & PacketCount:10 \\
                \hline
                Flag = ACK & Flag:ACK \\
                \hline
                Protocol = ICMP & Protocol:ICMP \\
                \hline
                Destination IP = 10.0.0.5 & DstIP:10.0.0.5 \\
                \hline
                \end{tabular}
            }}
        \end{table*}
        

    \section{Hash Embedding}
    Hash Embedding is a common lightweight feature encoding technology, which is particularly suitable for structured, high-dimensional, or large-number-of-categories network data. Its core approach is to convert each field name/field value (or a combination of the two) into a set of indexes through a hash function (such as MurmurHash3), and query the embedding table to obtain a fixed-length semantic vector.
    The main method is to combine the (field name, field value) of each data sample and pass it through a hash function such as MurmurHash3 to obtain a set of row/col indexes. This set of indices is then used to query a multi-dimensional embedding table, where an initial random, trainable semantic vector is stored at each position. The multi-field embedding vectors are concatenated (flattened) or aggregated, and then the data is given to the anomaly detection model for learning and inference.
    Weinberger et al. \cite{pang2021deep} proposed Feature Hashing to solve the coding efficiency problem of high-dimensional sparse data. L. Zhu et al. \cite{weinberger2009feature} combined Feature Hashing with a multi-layer perception for IoT intrusion anomaly detection and proved that it can significantly reduce the number of model parameters and improve computing efficiency. MurmurHash3 is widely used to replace traditional hashing techniques because of its uniform distribution, fast calculation, and cons.istency across languages.


    \section{Multi-Layer Perceptron in Anomaly Detection}
    Multi-Layer Perceptrons (MLPs) have been widely applied in the field of anomaly detection due to their capability to model non-linear relationships between input features and hidden patterns. Unlike traditional statistical models that rely on predefined thresholds or assumptions about data distribution, MLPs are capable of learning complex, high-dimensional feature representations in a data-driven manner \cite{lecun2015deep}.

    In recent years, MLP-based anomaly detection methods have been employed in various domains, including network security \cite{moustafa2019new}, industrial control systems \cite{kim2020cyber}, and IoT environments \cite{nguyen2020autoencoder}. These models typically consist of multiple fully connected layers with nonlinear activation functions, such as ReLU or sigmoid, enabling the learning of hierarchical semantic features. The outputs are used to distinguish between normal and abnormal behavior based on reconstruction error, classification scores, or learned distance metrics.

    While MLPs are not as expressive as deep convolutional or recurrent models, their low computational cost and ease of deployment make them particularly attractive for lightweight and real-time anomaly detection systems. In our work, we leverage an MLP-based encoder to transform hash-embedded feature vectors into semantic representations, which are then evaluated using Mahalanobis distance for effective anomaly scoring.


    \section{Semantic Vector}
    Semantic vector representations, originally popularized in natural language processing (NLP), have gained traction in anomaly detection tasks due to their ability to encode complex contextual information into fixed-length embeddings. In security-related applications, raw network traffic often contains heterogeneous features that lack explicit semantics; transforming these into semantic vectors enables better generalization and interpretability \cite{mikolov2013distributed}.

    Recent works have applied semantic encoding strategies, such as Word2Vec and sequence embeddings, to convert protocol names, IP addresses, or header fields into high-dimensional vectors \cite{shapira2021flow,li2020embedding}. These semantic vectors capture latent relationships between fields and behaviors, allowing downstream models to detect subtle deviations from normal patterns. For instance, Shapira et al. \cite{shapira2021flow} proposed Flow2Vec, which encodes sequences of network events into dense vectors, improving anomaly detection in encrypted traffic.

    Compared to one-hot encoding or manually crafted features, semantic vectors provide a richer and more scalable representation, particularly when combined with deep learning models. In this work, we construct semantic vectors from tokenized field-value pairs using a hash-based embedding scheme followed by an MLP encoder. This method ensures that semantic relationships among network features are preserved while maintaining computational efficiency.

   

\end{ZhChapter}