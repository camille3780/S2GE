\begin{ZhChapter}
    \chapter{Related Work}
    This section will introduce the relevant basic knowledge, including existing the IoT network intrusion detection methods, Tokenization, Hash Embedding, and language tags.

    \section{Network Intrusion Detection System in IoT}
    In recent years, the proliferation of Internet of Things (IoT) devices has led to an increased focus on developing effective network intrusion detection systems (NIDS) tailored to the specific characteristics of IoT environments. Various approaches have been proposed to address the challenges associated with high-volume, heterogeneous network traffic, constrained device capabilities, and evolving attack patterns. Kharoubi et al.~\cite{kharoubi2025nidscnn} proposed NIDS-DL-CNN, a convolutional neural network (CNN)-based detection system designed for IoT security. By applying CNN layers to extract spatial features from traffic data, the model achieved high classification performance on datasets such as CICIoT2023 and CICIoMT2024. The authors demonstrated that their method achieved excellent precision and recall in both binary and multi-class scenarios. However, a notable limitation of the CNN-based approach lies in its inability to fully capture temporal dependencies across packet sequences, and its reliance on supervised learning requires extensive labeled datasets. Ashraf et al.~\cite{ashraf2025inids} introduced a real-time intrusion detection system (NIDS) based on traditional machine learning classifiers applied to the BoT-IoT dataset. The study compared seven algorithms, including Random Forest, Artificial Neural Networks (ANN) et al., and Support Vector Machines. Their results showed that Random Forest and ANN achieved the highest accuracy and robustness among all tested classifiers. Despite its efficiency, the NIDS system was highly dependent on manual feature engineering and lacked adaptability to novel threats, which are critical in fast-evolving IoT environments. Elrawy et al.~\cite{elrawy2018survey} conducted a comprehensive survey of intrusion detection methodologies in IoT-based smart environments, categorizing techniques according to architectural design (centralized vs. distributed), detection strategy (signature-based, anomaly-based, or hybrid), and system layer (perception, network, application). While the survey provided valuable insights and synthesized a broad range of IDS approaches, it lacked implementation evidence and empirical comparisons, limiting its utility for practical system design. Collectively, these studies highlight the trade-offs between detection performance, computational cost, and deployment feasibility. Deep learning models offer strong accuracy but demand computational resources, while traditional classifiers provide efficiency but often lack flexibility.


    Table 2.1 summarizes the characteristics that are often used to identify abnormal or attack behaviors in Internet of Things (IoT) network traffic, and gives a concise explanation of these characteristics. For example, abnormal target port numbers, such as 22 (SSH), 23 (Telnet), 80 (HTTP), 443 (HTTPS), etc., are often the main targets of attackers. If a device attempts to connect to these ports in large numbers in a short period of time, it may show signs of port scanning, brute force cracking, or service denial of service attacks. Abnormal packet size, a large number of connected IPs in a short period of time, or unusual TCP flag bit combinations may reveal worm proliferation, DDoS attacks, or scanning behaviors.
    Next, abnormal changes in Flow Duration are often related to scanning activities or data theft. Too short connection time may be the result of automated scanning tools, while abnormally long connections may indicate that the attacker is conducting long-term data exfiltration or maintaining communication with the control server (C&C).
    Total Forward Packets and Flow Bytes per Second, two features related to traffic size and rate, can help reveal abnormal behaviors such as flood attacks, illegal data transmission, or packet spoofing. If the number of packets in a single direction is abnormally high or low, or the flow rate fluctuates dramatically, it may be a sign of a distributed denial of service (DDoS) attack or unauthorized data transmission.
    Packet Length can reflect abnormal patterns in packet content. For example, if an IoT device frequently sends fixed-length, abnormally long, or abnormally short packets, it may mean that malicious data has been implanted in the packet, or a spoofing and propagation attack is being carried out.
    In addition, Protocol Type is used to observe the use of protocols. Under normal circumstances, IoT devices mostly use TCP or UDP for communication; if a large number of ICMP (such as ping) or other unusual protocols appear, it may mean that the attacker is trying to exploit protocol weaknesses or bypass security protection mechanisms.
    The Source IP / Destination IP and Number of Connections features are related to the source and frequency of the connection. When a single source IP establishes a large number of connections in a short period of time, or a large number of new and unprecedented IP addresses appear, it often indicates that the network may be under scanning, DDoS attacks, or IP spoofing and other malicious behaviors.
    TCP Flags can reveal abnormal activities at the transport layer. Unusual flag combinations (such as SYN and FIN appearing at the same time, or a large number of RST packets in a short period of time) are often used in stealth scans or TCP protocol-related attacks, and are highly warning.
    Therefore, through these detailed feature analyses, the security system can more accurately grasp the traffic pattern and thus strengthen the protection of the IoT environment. Overall, this table not only summarizes the abnormal indicators commonly used in academia and industry, but also provides a specific reference for the subsequent construction of intrusion detection models.

    \begin{table*}[htbp]
        \centering
        \caption{Common Anomalous Features in IoT Network Traffic and Their Descriptions}
        \label{tab:iot_abnormal_features}
        \makebox[\linewidth][c]{
            \renewcommand\arraystretch{1.3}{
                \begin{tabular}{| l | c | p{8.5cm} |}
                    \hline
                    \textbf{Feature}           \textbf{Reference}                  & \textbf{Description}                                                                                                                                                                    \\
                    \hline
                    Destination Port            \cite{tang2016deep}                & Specific port targets (e.g., 22, 23, 80, 443) are often associated with attacks. Abnormal access to these ports may suggest behaviors such as scanning, DDoS, or brute-force intrusion. \\
                    \hline
                    Flow Duration               \cite{tharewal2022intrusion}       & Extremely short or long connection durations within brief timeframes may signal scanning activity or data exfiltration.                                                                 \\
                    \hline
                    Total Forward Packets       \cite{sharafaldin2018cicflowmeter} & Unusually high or low packet counts in one direction may indicate abnormal sessions or flooding behavior.                                                                               \\
                    \hline
                    Packet Length               \cite{tang2016deep}                & Anomalies in packet size—whether fixed, too long, or too short—often reflect malicious traffic like botnet propagation or worms.                                                        \\
                    \hline
                    Protocol Type               \cite{tharewal2022intrusion}       & Sudden increases in uncommon protocols (e.g., ICMP, UDP) may reveal attempts to exploit protocol vulnerabilities or bypass filters.                                                     \\
                    \hline
                    Source IP / Destination IP  \cite{tavallaee2009kdd}            & Repeated access from abnormal IP addresses, or sudden surges in novel IP sources, are indicative of scanning, spoofing, or DDoS activity.                                               \\
                    \hline
                    Flow Bytes per Second       \cite{tang2016deep}                & Sharp fluctuations—surges or drops—in flow byte rate may suggest DoS attacks or unauthorized data transfer.                                                                             \\
                    \hline
                    TCP Flags                  \cite{sharafaldin2018cicflowmeter}  & Unusual combinations (e.g., SYN, FIN, RST) can indicate stealth scans or TCP-based flooding.                                                                                            \\
                    \hline
                    Number of Connections       \cite{tang2016deep}                & A large number of new connections established by a single IP in a short time often reflects worm propagation or botnet coordination.                                                    \\
                    \hline
                \end{tabular}
            }}
    \end{table*}






    \section{Tokenization}
    Recent advances in IoT network security have explored various lightweight feature representations to enable efficient anomaly detection on constrained devices. One promising direction is the use of \textbf{token-based features}, which tokenize sequences from network traffic—such as packet headers, payloads, or session patterns—and analyze them using statistical or learning-based approaches.

    \cite{meidan2018n} proposed N-BaIoT, a deep autoencoder-based anomaly detection system, which tokenizes device-specific network flows and learns latent representations of normal behavior. Their approach segments raw traffic into fixed-length feature vectors using frequency and count-based tokenization over flows. In their evaluation on multiple IoT device datasets (e.g., Baby Monitor, Thermostat), N-BaIoT achieved an average detection accuracy of 99.8\% and false positive rates below 0.5\%.

    In a similar direction, \cite{rahman2020token} explored the application of token sequence modeling on IoT DNS and HTTP traffic. They treated URL paths and DNS queries as text sequences and applied n-gram tokenization, followed by TF-IDF vectorization. These features were then used to train classical ML classifiers (Random Forest, SVM), achieving F1-scores over 96\% on malicious domain detection tasks. Their results emphasize the high discriminative power of token frequency patterns.

    Furthermore, \cite{torres2022iotbert} introduced IoT-BERT, a transformer-based model pretrained on tokenized sequences of IoT network packets. Using self-supervised learning on tokenized payloads and headers, IoT-BERT learned context-aware embeddings of network behavior. In downstream tasks like anomaly classification and attack attribution, their model outperformed traditional CNN/RNN-based methods, achieving over 98\% accuracy on the TON\_IoT dataset.

    These studies highlight that token-based feature extraction offers a balance between computational efficiency and semantic richness, especially valuable in resource-limited IoT environments. Compared to raw packet-level analysis, tokenization enables scalable, interpretable modeling of sequential network behavior.




    \begin{table*}[htbp]
        \centering
        \caption{Examples of Field-Value Tokenization in IoT Network Traffic}
        \label{tab:tokenization_examples}
        \makebox[\linewidth][c]{
            \renewcommand\arraystretch{1.2}{
                \begin{tabular}{| l | l |}
                    \hline
                    \textbf{Feature Field}    \& \textbf{Tokenized Representation} \\
                    \hline
                    Protocol = TCP            \& Protocol:TCP                      \\
                    \hline
                    Destination Port = 443    \& DstPort:443                       \\
                    \hline
                    Source Port = 80          \& SrcPort:80                        \\
                    \hline
                    Source IP = 192.168.0.1   \& SrcIP:192.168.0.1                 \\
                    \hline
                    Flow Duration = 120000    \& FlowDuration:120000               \\
                    \hline
                    Payload Bytes = 56        \& PayloadBytes:56                   \\
                    \hline
                    Packet Count = 10         \& PacketCount:10                    \\
                    \hline
                    Flag = ACK                \& Flag:ACK                          \\
                    \hline
                    Protocol = ICMP           \& Protocol:ICMP                     \\
                    \hline
                    Destination IP = 10.0.0.5 \& DstIP:10.0.0.5                    \\
                    \hline
                \end{tabular}
            }}
    \end{table*}


    \section{Hash Embedding} \label{sec:hash_embedding}
    To address the challenges of high-dimensional categorical or textual data in IoT traffic (e.g., URLs, IPs, headers), recent research has explored \textbf{hash embedding} as an efficient representation technique. Hash embedding reduces memory usage and overfitting by mapping high-cardinality tokens into fixed-size low-dimensional vectors using multiple hash functions.

    \cite{gupta2020hash} proposed a hash embedding-based method for representing protocol-level IoT traffic, especially targeting categorical fields such as destination ports, device types, and payload signatures. Their approach utilized a multi-hash embedding layer before feeding data into a shallow neural network for anomaly detection. Experiments on the BoT-IoT dataset showed a 40\% reduction in model size while retaining over 97\% detection accuracy compared to one-hot encoding.

    \cite{feng2021lightweight} further integrated hash embeddings into a lightweight convolutional architecture for edge-based IoT security. Their model encoded domain names, user-agent strings, and API patterns using 2-way hash embeddings, which significantly reduced the input dimension and inference latency. They demonstrated that their system could run on resource-constrained devices (e.g., Raspberry Pi) with only 30ms per inference, while achieving an F1-score of 96.5\% on the CIC-ToN-IoT dataset.
    In another study, \cite{yin2022efficient} applied hash embeddings to convert netflow token sequences into compressed, learnable embeddings used in attention-based anomaly detection models. Their work highlights the robustness of hash embeddings in minimizing collision effects and handling unseen tokens during inference, which is particularly important in dynamic IoT networks.
    Overall, these studies confirm that hash embedding is a scalable and effective technique for representing sparse or categorical IoT traffic features, enabling fast and accurate detection of malicious behaviors under memory and computation constraints.
    To efficiently represent high-cardinality categorical data or sparse tokenized sequences in IoT traffic, the \textbf{hashing trick} (also known as feature hashing) has become a widely used method in machine learning and anomaly detection tasks. Unlike one-hot encoding, which results in extremely high-dimensional and sparse vectors, the hash trick maps input features into a lower-dimensional fixed-size vector using one or more hash functions.

    The basic idea is to apply a hash function $h(\cdot)$ to map each feature $x_i$ to an index in a fixed-size vector of length $d$. The value is then accumulated in the corresponding index, optionally with a sign function $g(\cdot)$ to preserve distribution balance. Formally, the transformation is defined as:

    \begin{equation}
        \mathbf{z}_j = \sum_{i: h(x_i) = j} g(x_i) \cdot v(x_i), \quad j = 1, 2, \dots, d
    \end{equation}

    where:
    \begin{itemize}
        \item $x_i$ is the $i$-th feature or token (e.g., a word, IP, URL fragment).
        \item $v(x_i)$ is the associated value (e.g., frequency or 1 for binary presence).
        \item $h(x_i) \in \{1, \dots, d\}$ is the hash function mapping to index $j$.
        \item $g(x_i) \in \{-1, +1\}$ is a second hash function that determines the sign to reduce collisions (optional).
        \item $\mathbf{z} \in \mathbb{R}^d$ is the final hash-embedded vector.
    \end{itemize}

    This approach has multiple benefits in the context of IoT:
    \begin{itemize}
        \item \textbf{Memory efficiency:} The output vector dimension $d$ is predefined, independent of vocabulary size.
        \item \textbf{Collision tolerance:} While hash collisions may occur, in practice they have minimal effect when $d$ is sufficiently large.
        \item \textbf{Online learning suitability:} Hash functions are fast and do not require a predefined dictionary, making them ideal for streaming IoT data.
    \end{itemize}

    Multiple-hash embedding schemes~\cite{hashingtrick} extend this idea by combining several independent hash functions and learnable embedding matrices to further mitigate collision effects and improve generalization.


    \section{Multi-Layer Perceptron in Anomaly Detection}
    Multi-Layer Perceptrons (MLPs) have been widely applied in the field of anomaly detection due to their capability to model non-linear relationships between input features and hidden patterns. Unlike traditional statistical models that rely on predefined thresholds or assumptions about data distribution, MLPs are capable of learning complex, high-dimensional feature representations in a data-driven manner \cite{lecun2015deep}.

    In recent years, MLP-based anomaly detection methods have been employed in various domains, including network security \cite{moustafa2019new}, industrial control systems \cite{kim2020cyber}, and IoT environments \cite{nguyen2020autoencoder}. These models typically consist of multiple fully connected layers with nonlinear activation functions, such as ReLU or sigmoid, enabling the learning of hierarchical semantic features. The outputs are used to distinguish between normal and abnormal behavior based on reconstruction error, classification scores, or learned distance metrics.

    While MLPs are not as expressive as deep convolutional or recurrent models, their low computational cost and ease of deployment make them particularly attractive for lightweight and real-time anomaly detection systems. In our work, we leverage an MLP-based encoder to transform hash-embedded feature vectors into semantic representations, which are then evaluated using Mahalanobis distance for effective anomaly scoring.


    \section{Semantic Vector}
    Semantic vector representations, originally popularized in natural language processing (NLP), have gained traction in anomaly detection tasks due to their ability to encode complex contextual information into fixed-length embeddings. In security-related applications, raw network traffic often contains heterogeneous features that lack explicit semantics; transforming these into semantic vectors enables better generalization and interpretability \cite{mikolov2013distributed}.

    Recent works have applied semantic encoding strategies, such as Word2Vec and sequence embeddings, to convert protocol names, IP addresses, or header fields into high-dimensional vectors \cite{shapira2021flow,li2020embedding}. These semantic vectors capture latent relationships between fields and behaviors, allowing downstream models to detect subtle deviations from normal patterns. For instance, Shapira et al. \cite{shapira2021flow} proposed Flow2Vec, which encodes sequences of network events into dense vectors, improving anomaly detection in encrypted traffic.

    Compared to one-hot encoding or manually crafted features, semantic vectors provide a richer and more scalable representation, particularly when combined with deep learning models. In this work, we construct semantic vectors from tokenized field-value pairs using a hash-based embedding scheme followed by an MLP encoder. This method ensures that semantic relationships among network features are preserved while maintaining computational efficiency.



\end{ZhChapter}