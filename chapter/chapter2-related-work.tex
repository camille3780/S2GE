\begin{ZhChapter}
    \chapter{Related Work}
    This section will introduce the relevant basic knowledge, including existing the IoT network intrusion detection methods, Tokenization, Hash Embedding, and language tags. 還在修改,可先跳過。

    \section{Network Intrusion Detection System in IoT}
    In recent years, the proliferation of Internet of Things (IoT) devices has led to an increased focus on developing effective network intrusion detection systems (NIDS) tailored to the specific characteristics of IoT environments. Various approaches have been proposed to address the challenges associated with high-volume, heterogeneous network traffic, constrained device capabilities, and evolving attack patterns. Kharoubi et al.~\cite{kharoubi2025nidscnn} proposed NIDS-DL-CNN, a convolutional neural network (CNN)-based detection system designed for IoT security. By applying CNN layers to extract spatial features from traffic data, the model achieved high classification performance on datasets such as CICIoT2023 and CICIoMT2024. The authors demonstrated that their method achieved excellent precision and recall in both binary and multi-class scenarios. However, a notable limitation of the CNN-based approach lies in its inability to fully capture temporal dependencies across packet sequences, and its reliance on supervised learning requires extensive labeled datasets. Ashraf et al.~\cite{ashraf2025inids} introduced a real-time intrusion detection system (NIDS) based on traditional machine learning classifiers applied to the BoT-IoT dataset. The study compared seven algorithms, including Random Forest, Artificial Neural Networks (ANN) et al., and Support Vector Machines. Their results showed that Random Forest and ANN achieved the highest accuracy and robustness among all tested classifiers. Despite its efficiency, the NIDS system was highly dependent on manual feature engineering and lacked adaptability to novel threats, which are critical in fast-evolving IoT environments. Elrawy et al.~\cite{elrawy2018survey} conducted a comprehensive survey of intrusion detection methodologies in IoT-based smart environments, categorizing techniques according to architectural design (centralized vs. distributed), detection strategy (signature-based, anomaly-based, or hybrid), and system layer (perception, network, application). While the survey provided valuable insights and synthesized a broad range of IDS approaches, it lacked implementation evidence and empirical comparisons, limiting its utility for practical system design. Collectively, these studies highlight the trade-offs between detection performance, computational cost, and deployment feasibility. Deep learning models offer strong accuracy but demand computational resources, while traditional classifiers provide efficiency but often lack flexibility.


    Table 2.1 summarize the eight characteristics commonly used by academia and industry are summarized below.

    First, the "target port" is one of the most direct attack indicators. IoT devices usually open a few ports for normal communication. If a large number of connections are initiated to common ports such as 22 (SSH), 23 (Telnet), 80 (HTTP), 443 (HTTPS) in a short period of time, it is very likely a precursor to malicious port scanning, brute force cracking or service denial of service attacks\cite{sicari2015}.

    Next, Packet Size and Flow Bytes per Second reflect the abnormal pattern of data flow. Oversized packets may contain malicious payloads, while undersized packets may be detection messages for reconnaissance; and drastically fluctuating packet numbers or byte frequencies are common in flood attacks or covert data leakage behaviors\cite{moustafa2015}.

    Number of Connections is used to observe source diversity. A large number of IP connections with different destinations initiated from a single source in a very short period of time often means that the scanning tool is probing the network segment; on the contrary, if a large number of new and never-before-seen IPs appear, it may be caused by a distributed denial of service (DDoS) attack or IP spoofing\cite{sicari2015}.

    Flow Duration is also critical. Scanning tools often use very short connections to detect various services; and if there is a long continuous connection, it means that the attacker is performing a long-term data exfiltration or maintaining communication with the control server (C\&C) to instigate deeper penetration\cite{moustafa2015}.

    Protocol Type and TCP Flags help to further interpret the behavior pattern. IoT devices generally operate mainly based on TCP/UDP. If a large number of ICMP (such as ping) or other rare protocols appear, it means that the attacker may be exploiting protocol weaknesses or trying to circumvent security protection. In addition, abnormal flag combinations such as the simultaneous appearance of SYN and FIN, or a large number of RST packets in a short period of time, are typical features of stealth scanning or specific TCP attacks\cite{moustafa2015}.

    Finally, "Packet Length" can reflect more subtle anomalies: when a device repeatedly sends packets of a fixed length that are too long or too short, it is very likely that the attacker has implanted malicious data in it, or is performing a forgery and spreading attack.

    Combining the above eight features, the security system can detect and intercept various IoT attack behaviors at the earliest stage, providing a solid basis for the construction of subsequent intrusion detection models.

    \begin{table*}[htbp]
        \centering
        \caption{Common Anomalous Features in IoT Network Traffic and Their Descriptions}
        \label{tab:iot_abnormal_features}
        \makebox[\linewidth][c]{
            \renewcommand\arraystretch{1.3}{
                \begin{tabular}{| l | p{10cm} |}
                    \hline
                    \textbf{Feature}                                        & \textbf{Description}                                                            \\
                    \hline
                    Destination Port\cite{tang2016deep}                     &
                    Specific port targets (e.g., 22, 23, 80, 443) are often associated with attacks.
                    Abnormal access to these ports may suggest behaviors such as scanning, DDoS, or brute-force intrusion.                                    \\
                    \hline
                    Flow Duration\cite{tharewal2022intrusion}               &
                    Extremely short or long connection durations within brief timeframes may signal scanning activity or data exfiltration.                   \\
                    \hline
                    Total Forward Packets\cite{sharafaldin2018cicflowmeter} &
                    Unusually high or low packet counts in one direction may indicate abnormal sessions or flooding behavior.                                 \\
                    \hline
                    Packet Length\cite{tang2016deep}                        &
                    Anomalies in packet size—whether fixed, too long, or too short—often reflect malicious traffic like botnet propagation or worms.          \\
                    \hline
                    Protocol Type\cite{tharewal2022intrusion}               &
                    Sudden increases in uncommon protocols (e.g., ICMP, UDP) may reveal attempts to exploit protocol vulnerabilities or bypass filters.       \\
                    \hline
                    Source IP / Destination IP\cite{tavallaee2009kdd}       &
                    Repeated access from abnormal IP addresses, or sudden surges in novel IP sources, are indicative of scanning, spoofing, or DDoS activity. \\
                    \hline
                    Flow Bytes per Second\cite{tang2016deep}                &
                    Sharp fluctuations—surges or drops—in flow byte rate may suggest DoS attacks or unauthorized data transfer.                               \\
                    \hline
                    TCP Flags\cite{sharafaldin2018cicflowmeter}             &
                    Unusual combinations (e.g., SYN, FIN, RST) can indicate stealth scans or TCP-based flooding.                                              \\
                    \hline
                    Number of Connections\cite{tang2016deep}                &
                    A large number of new connections established by a single IP in a short time often reflects worm propagation or botnet coordination.      \\
                    \hline
                \end{tabular}
            }
        }
    \end{table*}







    \section{Tokenization}
    \noindent Recent advances in IoT network security have explored various lightweight feature representations to enable efficient anomaly detection on constrained devices. One promising direction is the use of \textbf{token-based features}, which tokenize sequences from network traffic—such as packet headers, payloads, or session patterns—and analyze them using statistical or learning-based approaches.


    \noindent In 2018, Yair Meidan, Michael Bohadana, Yael Mathov, Yisroel Mirsky, Dominik Breitenbacher, Asaf Shabtai, Yuval Elovici et al. published a paper \emph{“N-BaIoT: Network-based Detection of IoT Botnet Attacks Using Deep Autoencoders”} in IEEE Pervasive Computing, proposing an anomaly detection system based on deep autoencoders. They divided the network traffic of different IoT devices into feature vectors of fixed length and tokenized them based on frequency and count to learn the latent representation of normal behavior, achieving a detection accuracy of 99.8% and a false alarm rate of less than 0.5% on multiple sets of devices (such as Baby Monitor and Thermostat). \cite{meidan2018n}


    \noindent Next, in 2020, A. H. M. Rahman, B. K. Roy, and C. Li, in their paper
    \emph{“Token Sequence Modeling for Anomaly Detection in IoT Network Traffic”}, treated URL paths and DNS queries as text sequences, used n-gram segmentation and then TF-IDF vectorization, and classified them using Random Forest and Support Vector Machine (SVM), achieving an F1 score of over 96\% in the malicious domain name detection task. \cite{rahman2020token}


    \noindent In addition, Miguel Torres, Isabel Rojas, and Carlos Martinez published a paper in 2022,
    \emph{“IoT-BERT: Pretraining Transformers for IoT Network Packet Sequences”}, which introduced a Transformer model called IoT-BERT, which uses self-supervised learning to pre-train on tokenized packet headers and payloads, learns context-aware network behavior embeddings, and achieves over 98\% accuracy in anomaly classification and attack attribution tasks on the TON\_IoT dataset. \cite{torres2022iotbert}
    Table 2.2 還沒補說明


    \begin{table*}[htbp]
        \centering
        \caption{Examples of Field-Value Tokenization in IoT Network Traffic}
        \label{tab:tokenization_examples}
        \makebox[\linewidth][c]{
            \renewcommand\arraystretch{1.2}{
                \begin{tabular}{| l | p{10cm} |}
                    \hline
                    \textbf{Feature Field}    & \textbf{Tokenized Representation} \\
                    \hline
                    Protocol = TCP            & Protocol:TCP                      \\
                    \hline
                    Destination Port = 443    & DstPort:443                       \\
                    \hline
                    Source Port = 80          & SrcPort:80                        \\
                    \hline
                    Source IP = 192.168.0.1   & SrcIP:192.168.0.1                 \\
                    \hline
                    Flow Duration = 120000    & FlowDuration:120000               \\
                    \hline
                    Payload Bytes = 56        & PayloadBytes:56                   \\
                    \hline
                    Packet Count = 10         & PacketCount:10                    \\
                    \hline
                    Flag = ACK                & Flag:ACK                          \\
                    \hline
                    Protocol = ICMP           & Protocol:ICMP                     \\
                    \hline
                    Destination IP = 10.0.0.5 & DstIP:10.0.0.5                    \\
                    \hline
                \end{tabular}
            }
        }
    \end{table*}



    \section{Hash Embedding} \label{sec:hash_embedding}
    To address the challenges of high-dimensional categorical or textual data in IoT traffic (e.g., URLs, IPs, headers), recent research has explored \textbf{hash embedding} as an efficient representation technique. Hash embedding reduces memory usage and overfitting by mapping high-cardinality tokens into fixed-size low-dimensional vectors using multiple hash functions.

    \cite{gupta2020hash} proposed a hash embedding-based method for representing protocol-level IoT traffic, especially targeting categorical fields such as destination ports, device types, and payload signatures. Their approach utilized a multi-hash embedding layer before feeding data into a shallow neural network for anomaly detection. Experiments on the BoT-IoT dataset showed a 40\% reduction in model size while retaining over 97\% detection accuracy compared to one-hot encoding.

    \cite{feng2021lightweight} further integrated hash embeddings into a lightweight convolutional architecture for edge-based IoT security. Their model encoded domain names, user-agent strings, and API patterns using 2-way hash embeddings, which significantly reduced the input dimension and inference latency. They demonstrated that their system could run on resource-constrained devices (e.g., Raspberry Pi) with only 30ms per inference, while achieving an F1-score of 96.5\% on the CIC-ToN-IoT dataset.
    In another study, \cite{yin2022efficient} applied hash embeddings to convert netflow token sequences into compressed, learnable embeddings used in attention-based anomaly detection models. Their work highlights the robustness of hash embeddings in minimizing collision effects and handling unseen tokens during inference, which is particularly important in dynamic IoT networks.
    Overall, these studies confirm that hash embedding is a scalable and effective technique for representing sparse or categorical IoT traffic features, enabling fast and accurate detection of malicious behaviors under memory and computation constraints.
    To efficiently represent high-cardinality categorical data or sparse tokenized sequences in IoT traffic, the \textbf{hashing trick} (also known as feature hashing) has become a widely used method in machine learning and anomaly detection tasks. Unlike one-hot encoding, which results in extremely high-dimensional and sparse vectors, the hash trick maps input features into a lower-dimensional fixed-size vector using one or more hash functions.

    The basic idea is to apply a hash function $h(\cdot)$ to map each feature $x_i$ to an index in a fixed-size vector of length $d$. The value is then accumulated in the corresponding index, optionally with a sign function $g(\cdot)$ to preserve distribution balance. Formally, the transformation is defined as:

    \begin{equation}
        \mathbf{z}_j = \sum_{i: h(x_i) = j} g(x_i) \cdot v(x_i), \quad j = 1, 2, \dots, d
    \end{equation}

    where:
    \begin{itemize}
        \item $x_i$ is the $i$-th feature or token (e.g., a word, IP, URL fragment).
        \item $v(x_i)$ is the associated value (e.g., frequency or 1 for binary presence).
        \item $h(x_i) \in \{1, \dots, d\}$ is the hash function mapping to index $j$.
        \item $g(x_i) \in \{-1, +1\}$ is a second hash function that determines the sign to reduce collisions (optional).
        \item $\mathbf{z} \in \mathbb{R}^d$ is the final hash-embedded vector.
    \end{itemize}

    This approach has multiple benefits in the context of IoT:
    \begin{itemize}
        \item \textbf{Memory efficiency:} The output vector dimension $d$ is predefined, independent of vocabulary size.
        \item \textbf{Collision tolerance:} While hash collisions may occur, in practice they have minimal effect when $d$ is sufficiently large.
        \item \textbf{Online learning suitability:} Hash functions are fast and do not require a predefined dictionary, making them ideal for streaming IoT data.
    \end{itemize}

    Multiple-hash embedding schemes~\cite{hashingtrick} extend this idea by combining several independent hash functions and learnable embedding matrices to further mitigate collision effects and improve generalization.


    \section{Multi-Layer Perceptron in Anomaly Detection} 在這裡說明技術,放圖
    Multi-Layer Perceptrons (MLPs) have been widely applied in the field of anomaly detection due to their capability to model non-linear relationships between input features and hidden patterns. Unlike traditional statistical models that rely on predefined thresholds or assumptions about data distribution, MLPs are capable of learning complex, high-dimensional feature representations in a data-driven manner \cite{lecun2015deep}.

    In recent years, MLP-based anomaly detection methods have been employed in various domains, including network security \cite{moustafa2019new}, industrial control systems \cite{kim2020cyber}, and IoT environments \cite{nguyen2020autoencoder}. These models typically consist of multiple fully connected layers with nonlinear activation functions, such as ReLU or sigmoid, enabling the learning of hierarchical semantic features. The outputs are used to distinguish between normal and abnormal behavior based on reconstruction error, classification scores, or learned distance metrics.

    While MLPs are not as expressive as deep convolutional or recurrent models, their low computational cost and ease of deployment make them particularly attractive for lightweight and real-time anomaly detection systems. In our work, we leverage an MLP-based encoder to transform hash-embedded feature vectors into semantic representations, which are then evaluated using Mahalanobis distance for effective anomaly scoring.

    The Multi-Layer Perceptron (MLP) is one of the foundational deep learning architectures, widely used in classification and anomaly detection tasks due to its simplicity and capability to model non-linear decision boundaries. In the context of IoT network traffic analysis, MLP has been adopted as a lightweight yet powerful model for detecting abnormal behaviors across various device and protocol types.

    \cite{shone2018deep} introduced a hybrid deep learning approach combining a stacked autoencoder with an MLP classifier to detect network intrusions. Their model was evaluated on the NSL-KDD dataset, achieving an accuracy of 85.42\% and demonstrating superior performance over classical ML algorithms such as decision trees and SVM.

    Similarly, \cite{rahman2020deep} applied a pure MLP-based architecture for anomaly detection in the BoT-IoT dataset. The network consisted of three hidden layers with ReLU activation and dropout regularization. The results showed that MLP achieved over 98.5\% detection accuracy and maintained a false positive rate below 1\%, outperforming traditional algorithms such as KNN and Naive Bayes.

    \cite{javaid2016deep} further explored MLP in a deep learning pipeline tailored for IoT environments. They emphasized the importance of feature normalization and used a softmax output layer for multi-class classification. Their experiments on KDDCup'99 and UNSW-NB15 datasets revealed that MLP models trained on optimized features could achieve both high recall and precision in detecting diverse attack types, including DoS, probing, and user-to-root exploits.

    Despite its effectiveness, MLP has certain limitations. It lacks spatial or temporal awareness, making it less suitable for sequential data unless combined with other architectures (e.g., LSTM or CNN). However, for static, tabular representations of network flows, MLP remains a competitive choice due to its fast inference and low memory footprint, which are critical in real-time IoT security deployments.

    These findings suggest that MLP can serve as a strong baseline model in IoT anomaly detection pipelines, especially when combined with proper feature engineering and regularization techniques.

    Given an input vector $\mathbf{x} \in \mathbb{R}^d$, the computation through an MLP with $L$ hidden layers can be described as:

    \begin{equation}
        \mathbf{h}^{(1)} = \sigma\left( \mathbf{W}^{(1)} \mathbf{x} + \mathbf{b}^{(1)} \right)
        \label{eq:mlp_layer1}
    \end{equation}

    \begin{equation}
        \mathbf{h}^{(l)} = \sigma\left( \mathbf{W}^{(l)} \mathbf{h}^{(l-1)} + \mathbf{b}^{(l)} \right), \quad \text{for } l=2, \dots, L
        \label{eq:mlp_hidden}
    \end{equation}

    \begin{equation}
        \hat{\mathbf{y}} = f\left( \mathbf{W}^{(L+1)} \mathbf{h}^{(L)} + \mathbf{b}^{(L+1)} \right)
        \label{eq:mlp_output}
    \end{equation}

    Where:
    \begin{itemize}
        \item $\mathbf{W}^{(l)}$ and $\mathbf{b}^{(l)}$ are the weight matrix and bias vector of layer $l$
        \item $\sigma(\cdot)$ is a non-linear activation function (e.g., ReLU)
        \item $f(\cdot)$ is the output activation function (e.g., sigmoid or softmax)
        \item $\hat{\mathbf{y}}$ is the predicted output vector
    \end{itemize}

    \subsection{Activation Functions}

    Common activation functions include:

    \begin{itemize}
        \item \textbf{ReLU:} $\sigma(z) = \max(0, z)$
        \item \textbf{Sigmoid:} $\sigma(z) = \frac{1}{1 + e^{-z}}$
        \item \textbf{Tanh:} $\sigma(z) = \tanh(z) = \frac{e^z - e^{-z}}{e^z + e^{-z}}$
    \end{itemize}

    \subsection{Loss Functions}

    For binary classification, the binary cross-entropy loss is used:

    \begin{equation}
        \mathcal{L}_{\text{binary}} = - \frac{1}{N} \sum_{i=1}^{N} \left[ y_i \log(\hat{y}_i) + (1 - y_i) \log(1 - \hat{y}_i) \right]
        \label{eq:binary_loss}
    \end{equation}

    For multi-class classification with $C$ classes, the categorical cross-entropy is used:

    \begin{equation}
        \mathcal{L}_{\text{categorical}} = - \frac{1}{N} \sum_{i=1}^{N} \sum_{c=1}^{C} y_{i,c} \log(\hat{y}_{i,c})
        \label{eq:categorical_loss}
    \end{equation}

    \subsection{Backpropagation and Gradient Computation}

    Training is performed via gradient-based optimization (e.g., SGD, Adam), using backpropagation to compute gradients.

    Gradient of loss with respect to weights in layer $l$:

    \begin{equation}
        \frac{\partial \mathcal{L}}{\partial \mathbf{W}^{(l)}} = \delta^{(l)} \cdot \left( \mathbf{h}^{(l-1)} \right)^\top
        \label{eq:gradient_weight}
    \end{equation}

    Recursive definition of error term:

    \begin{equation}
        \delta^{(l)} = \left( \mathbf{W}^{(l+1)\top} \delta^{(l+1)} \right) \circ \sigma'\left( \mathbf{z}^{(l)} \right)
        \label{eq:delta_recursive}
    \end{equation}

    Where $\circ$ denotes element-wise multiplication, and $\sigma'(\cdot)$ is the derivative of the activation function.


    \section{Semantic Vector}
    Semantic vector representations, originally popularized in natural language processing (NLP), have gained traction in anomaly detection tasks due to their ability to encode complex contextual information into fixed-length embeddings. In security-related applications, raw network traffic often contains heterogeneous features that lack explicit semantics; transforming these into semantic vectors enables better generalization and interpretability \cite{mikolov2013distributed}.

    Recent works have applied semantic encoding strategies, such as Word2Vec and sequence embeddings, to convert protocol names, IP addresses, or header fields into high-dimensional vectors \cite{shapira2021flow,li2020embedding}. These semantic vectors capture latent relationships between fields and behaviors, allowing downstream models to detect subtle deviations from normal patterns. For instance, Shapira et al. \cite{shapira2021flow} proposed Flow2Vec, which encodes sequences of network events into dense vectors, improving anomaly detection in encrypted traffic.

    Compared to one-hot encoding or manually crafted features, semantic vectors provide a richer and more scalable representation, particularly when combined with deep learning models. In this work, we construct semantic vectors from tokenized field-value pairs using a hash-based embedding scheme followed by an MLP encoder. This method ensures that semantic relationships among network features are preserved while maintaining computational efficiency.



\end{ZhChapter}