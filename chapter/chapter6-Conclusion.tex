\begin{ZhChapter}

    \chapter{Conclusion and Future Work}
    This paper proposes an anomaly detection framework called S2GE-NIDS, which combines structured semantics and generation embedding technology to perform efficient and explainable anomaly detection for Internet of Things (IoT) network traffic. Experimental results show that this method can effectively capture complex semantic associations through double hash embedding and lightweight multi-layer perceptron (MLP) architecture on the public benchmark dataset CICIoT2024, and use Mahalanobis distance to score anomalies, achieving better precision and recall than existing classic models (such as Isolation Forest, One-Class SVM and AutoEncoder). At the same time, it has significant advantages in computing resource consumption and is suitable for resource-limited edge computing devices.

    The contributions of this study include the innovative combination of double hashing and linked lists to reduce hash collisions and effectively control the size of the embedding table, thereby improving space and time efficiency. The statistical judgment mechanism based on Mahalanobis distance not only improves the accuracy of anomaly detection, but also improves the interpretability of the model. And adopting a lightweight MLP structure to reduce the computational burden of deep models, it is suitable for real-time monitoring and embedded system deployment.

    This paper hopes to continue to improve and further optimize the model in the future to support real-time anomaly detection with high throughput and low latency. Enhanced anti-adversarial resistance: In the face of intelligent attackers using adversarial samples to evade detection, strategies such as adversarial training and adaptive adjustment of anomaly thresholds can be introduced in the future to enhance model resilience. Cross-domain generalization capability: The specific protocols and behavior patterns of different IoT scenarios vary. Future research can explore how to make the S2GE architecture more adaptable across domains. Development of interpretability and visualization tools: In order to support the decision-making of information security analysts, more complete interpretability mechanisms and visualization interfaces will be developed in the future to improve the understandability of abnormal events.

    In summary, this study provides an innovative architecture for IoT anomaly detection that combines structured semantics and generative embedding, showing good experimental results and application potential. Through subsequent research on the optimization of scalability, resilience and interpretability, it is expected to promote IoT security protection technology into a higher level of practical application.

\end{ZhChapter}