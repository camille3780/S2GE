\begin{ZhChapter}

    \chapter{Conclusion and Future Work}
    This paper proposes an anomaly detection framework called S2GE-NIDS, which combines structured semantics and generation embedding technology to perform efficient and explainable anomaly detection for Internet of Things (IoT) network traffic. Experimental results show that this method can effectively capture complex semantic associations through double hash embedding and lightweight multi-layer perceptron (MLP) architecture on the public benchmark dataset CICIoT2024, and use Mahalanobis distance to score anomalies, achieving better precision and recall than existing classic models (such as Isolation Forest, One-Class SVM and AutoEncoder). At the same time, it has significant advantages in computing resource consumption and is suitable for resource-limited edge computing devices.

    The contributions of this study include the innovative combination of double hashing and linked lists to reduce hash collisions and effectively control the size of the embedding table, thereby improving space and time efficiency. The statistical judgment mechanism based on Mahalanobis distance not only improves the accuracy of anomaly detection, but also improves the interpretability of the model. And adopting a lightweight MLP structure to reduce the computational burden of deep models, it is suitable for real-time monitoring and embedded system deployment.

    In future work, we plan to extend the current model by integrating additional sensor data and network traffic features, such as time-series data and behavioral patterns. By incorporating these multi-dimensional data sources, the model’s ability to detect complex anomalies can be enhanced, leading to improved generalization across diverse IoT environments.

    Furthermore, we aim to develop a real-time anomaly detection system that can be deployed on edge devices. This will involve optimizing the inference efficiency and minimizing latency to meet the stringent requirements of resource-constrained environments. The focus will be on designing lightweight models and leveraging hardware acceleration techniques to enable timely and accurate anomaly detection in operational settings.

    In summary, this study provides an innovative architecture for IoT anomaly detection that combines structured semantics and generative embedding, showing good experimental results and application potential. Through subsequent research on the optimization of scalability, resilience and interpretability, it is expected to promote IoT security protection technology into a higher level of practical application.

\end{ZhChapter}