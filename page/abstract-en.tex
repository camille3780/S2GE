% 英文摘要頁
\begin{EnAbstract}
    \begin{EnAbstractItems}
        % 論文名稱,請在 ntut-labels.tex 定義
        %\noindent \text Title: \eTitle

        % 論文總頁數,同最後一頁阿拉伯數字頁碼
        % Page number, same as the page number of the last page, not the total pages in pdf file.
        %\noindent \text Pages: (Fill it)

        % 校所別,請在 ntut-labels.tex 定義
        %\noindent \text School: \univEname

        % 系所別,請在 ntut-labels.tex 定義
        %\noindent \text Department: \deptEname

        % 畢業時間,請在 ntut-labels.tex 定義
        %\noindent \text Time: \eMonth, \eYear

        % 學位,請在 ntut-labels.tex 定義
        %\noindent \text Degree: \degreeEname

        % 研究生,請在 ntut-labels.tex 定義
        %\noindent \text Researcher: \myEname

        % 指導教授,請在 ntut-labels.tex 定義
        %\noindent \text Advisor: \advisorEname

        % 關鍵詞,請自己填,多個關鍵字以逗號 "," 隔開
        \noindent \text Keyword: IoT Security, Information Security, Anomaly Detection, Multilayer Perceptron, Semantic Vector

    \end{EnAbstractItems}

    \begin{EnAbstractDescription}
        As network environments become increasingly complex and dynamic, traditional intrusion detection methods struggle to keep pace with evolving threats and high-volume traffic. This paper proposes an efficient anomaly detection framework that leverages hash-based token embedding and a lightweight multi-layer perceptron (MLP) for the semantic representation of network flows. By transforming feature values into semantic tokens and utilizing a hashing trick for embedding lookup, our approach enables scalable and robust processing without maintaining an explicit vocabulary. The resulting embedding vectors are flattened and processed by the MLP to produce semantic vectors, which are clustered using a center loss strategy for unsupervised anomaly detection. Experimental results on the public CICIoT2024 benchmark dataset, comprising 238,687 samples and 10 network traffic features, demonstrate that our method achieves competitive accuracy with significantly improved computational efficiency compared to traditional attention-based models. Specifically, the model attains an F1-score of 0.88 and an Area Under Curve (AUC) of 0.98, with inference times of 0.8 milliseconds per sample, outperforming baseline methods such as Isolation Forest, One-Class SVM, and AutoEncoder in both detection accuracy and resource consumption. This lightweight and interpretable framework is well-suited for deployment in resource-constrained IoT environments.

    \end{EnAbstractDescription}

\end{EnAbstract}

