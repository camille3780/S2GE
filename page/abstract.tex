% 中文摘要頁
\begin{ZhAbstract}
    \begin{ZhAbstractItems}
        % 論文名稱,請在 ntut-labels.tex 定義
        \noindent \text 論文名稱:\cTitle

        % 論文頁數,請自己填
        \noindent \text 頁數:(請自己填)頁

        % 校所別,請在 ntut-labels.tex 定義
        \noindent \text 校所別:\univCname \space \deptCname

        % 畢業時間,請在 ntut-labels.tex 定義
        \noindent \text 畢業時間:\cAcademicYear 學年度 \space 第\cGraduateSemester 學期

        % 學位,請在 ntut-labels.tex 定義
        \noindent \text 學位:\degreeCname

        % 研究生,請在 ntut-labels.tex 定義
        \noindent \text 研究生:\myCname

        % 指導教授,請在 ntut-labels.tex 定義
        \noindent \text 指導教授:\advisorCname

        % 關鍵詞,請自己填,請自己填,多個關鍵字以逗號(、)隔開
        \noindent \text 關鍵詞:(請自己填)

    \end{ZhAbstractItems}

    \begin{ZhAbstractDescription}

        隨著網路環境日益複雜和動態化,傳統的入侵偵測方法難以跟上不斷演變的威脅。本文提出了一種高效的異常檢測框架,該框架利用基於哈希的令牌嵌入和輕量級多層感知器 (MLP) 對網路流進行語義表示。透過將特徵值轉換為語義向量,並利用雙重雜湊技巧進行嵌入查找,我們的方法無需維護詞彙表即可實現可擴展的處理。得到的的嵌入向量被展平並由 MLP 處理以產生語義向量,這些語義向量使用中心損失策略進行聚類,以進行無監督異常檢測。在包含 238,687 個樣本和 10 個網路流量特徵的公共 CICIoT2024 基準資料集上的實驗結果表明,檢測準確率和資源消耗方面均優於孤立森林、單類支持向量機和自編碼器等基準方法及傳統的模型,並且計算效率顯著提升。具體而言,該模型的 F1 得分為 0.88,曲線下面積為 0.98,每個樣本的推理時間為 0.8 毫秒,。因此這個輕量級且易於解釋的框架非常適合在資源受限的物聯網環境中部署。
    \end{ZhAbstractDescription}

\end{ZhAbstract}

