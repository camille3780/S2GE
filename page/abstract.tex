% 中文摘要頁
\begin{ZhAbstract}
    \begin{ZhAbstractItems}
        % 論文名稱,請在 ntut-labels.tex 定義
        \noindent \text 論文名稱:\cTitle

        % 論文頁數,請自己填
        \noindent \text 頁數:(請自己填)頁

        % 校所別,請在 ntut-labels.tex 定義
        \noindent \text 校所別:\univCname \space \deptCname

        % 畢業時間,請在 ntut-labels.tex 定義
        \noindent \text 畢業時間:\cAcademicYear 學年度 \space 第\cGraduateSemester 學期

        % 學位,請在 ntut-labels.tex 定義
        \noindent \text 學位:\degreeCname

        % 研究生,請在 ntut-labels.tex 定義
        \noindent \text 研究生:\myCname

        % 指導教授,請在 ntut-labels.tex 定義
        \noindent \text 指導教授:\advisorCname

        % 關鍵詞,請自己填,請自己填,多個關鍵字以逗號(、)隔開
        \noindent \text 關鍵詞:(請自己填)

    \end{ZhAbstractItems}

    \begin{ZhAbstractDescription}
        當前網路環境日益複雜且高度動態,傳統的入侵偵測方法已難以因應不斷演進的攻擊手法與龐大的資料流量。為此,本文提出一種高效的異常偵測框架,結合了雜湊式語意嵌入技術與輕量化多層感知器(MLP),用以表徵網路流量的語意特徵。本方法將特徵值轉換為語意標記(semantic token),並透過雜湊技巧完成嵌入查詢,無需維護龐大的詞彙表,即可實現具可擴展性且穩健的處理流程。接著,這些嵌入向量會被平坦化(flatten),並輸入至 MLP 模型以產生語意向量,進而透過中心損失(center loss)策略進行無監督異常偵測。實驗結果顯示,相較於傳統以注意力機制為核心的模型,本方法在準確率表現相當的情況下,大幅提升了運算效率與資源利用率。
    \end{ZhAbstractDescription}
    
\end{ZhAbstract}

