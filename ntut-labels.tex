%
% this file is encoded in utf-8
%

%% 這些設定值將會用於呈現在首頁上,請進行填入
%% Please fill in the information which will be shown on the cover and in the abstract. All Chinese and English information must be matched to each other. If you don’t have any Chinese information, please skip the Chinese ones, but all English Information is required.

%
% 中文論文設定值,請根據以下的範例進行填入
%

% 論文題目(中文)
% Thesis Title (Chinese)
% \newcommand\cTitle{待訂}

% 我的姓名(中文)
% My Name(Chinese)
\newcommand\myCname{周子榆}

% 指導教授的姓名 (中文),使用頓號隔開 
% Advisor (Chinese), use “、” to separate names
\newcommand\advisorCname{陳香君 博士}

% 校名 (中文)
% School Name(Chinese)
\newcommand\univCname{國立臺北科技大學}

% 系所名 (中文)
% Department Name(Chinese)
\newcommand\deptCname{資訊工程系碩士班}

% 學位名 (中文)
% Degree Name(Chinese)
\newcommand\degreeCname{碩士}

% 口試年份 (中文、民國)
% Year of Oral Defense(Chinese)
\newcommand\cYear{一百一十四}

% 口試月份 (中文)
% Month of Oral Defense(Chinese)
\newcommand\cMonth{五}

% 畢業學年度 (中文)
% 如 112 學年度第2學期畢業,當時為民國113年6月,學年度即為112,不是113。
\newcommand\cAcademicYear{一百一十三}

% 畢業學期(中文)
% Academic Year (Chinese)
\newcommand\cGraduateSemester{二}

%
% 英文論文設定值,請根據以下的範例進行填入
%

% 論文題目 (英文)
% Thesis Title (Engslih)
\newcommand\eTitle{Use Efficiency-based Genetic Programming to Create Loss Function for Image Classification Tasks}

% 我的姓名(英文)
% My Name(English)
\newcommand\myEname{Tzu-Yu Chou}

% 指導教授的姓名 (英文),使用逗號隔開
% 例如:Dr. Kuo Jong-Yi, Dr. A B-C, ...
%
% Advisor (Ensligh), use “, ” to separate names
% e.g. Dr. Kuo Jong-Yi, Dr. A B-C, ...
\newcommand\advisorEname{Shiang-Jiun Chen, Ph.D.}

% 校名(英文)
% School Name(English)
\newcommand\univEname{National Taipei University of Technology}

% 系所全名 (英文)
% Department Name(English)
\newcommand\fulldeptEname{Department of Computer Science and Information Engineering}

\newcommand\deptEname{Computer Science and Information Engineering}

% 學位名 (英文)
% Degree(English)
\newcommand\degreeEname{Master of Science}

% 口試年份 (阿拉伯數字、西元)
% Year of Oral Defense(English)
\newcommand\eYear{2025}

% 口試月份 (英文)
% Month of Oral Defense(English)
\newcommand\eMonth{May}

%畢業級別;用於書背列印;若無此需要可忽略
\newcommand\GraduationClass{113}

%%%%%%%%%%%%%%%%%%%%%%